%%
%% This is file `sample-manuscript.tex',
%% generated with the docstrip utility.
%%
%% The original source files were:
%%
%% samples.dtx  (with options: `manuscript')
%% 
%% IMPORTANT NOTICE:
%% 
%% For the copyright see the source file.
%% 
%% Any modified versions of this file must be renamed
%% with new filenames distinct from sample-manuscript.tex.
%% 
%% For distribution of the original source see the terms
%% for copying and modification in the file samples.dtx.
%% 
%% This generated file may be distributed as long as the
%% original source files, as listed above, are part of the
%% same distribution. (The sources need not necessarily be
%% in the same archive or directory.)
%%
%% Commands for TeXCount
%TC:macro \cite [option:text,text]
%TC:macro \citep [option:text,text]
%TC:macro \citet [option:text,text]
%TC:envir table 0 1
%TC:envir table* 0 1
%TC:envir tabular [ignore] word
%TC:envir displaymath 0 word
%TC:envir math 0 word
%TC:envir comment 0 0
%%
%%
%% The first command in your LaTeX source must be the \documentclass command.
\documentclass[sigconf,review,anonymous]{acmart}

%%
%% \BibTeX command to typeset BibTeX logo in the docs
\AtBeginDocument{%
  \providecommand\BibTeX{{%
    \normalfont B\kern-0.5em{\scshape i\kern-0.25em b}\kern-0.8em\TeX}}}

%% Rights management information.  This information is sent to you
%% when you complete the rights form.  These commands have SAMPLE
%% values in them; it is your responsibility as an author to replace
%% the commands and values with those provided to you when you
%% complete the rights form.
%\setcopyright{acmcopyright}
%\copyrightyear{2018}
%\acmYear{2018}
%\acmDOI{XXXXXXX.XXXXXXX}

%% These commands are for a PROCEEDINGS abstract or paper.
\acmConference[ICSE 2024]{46th International Conference on Software Engineering}{April 2024}{Lisbon, Portugal}


%%
%% Submission ID.
%% Use this when submitting an article to a sponsored event. You'll
%% receive a unique submission ID from the organizers
%% of the event, and this ID should be used as the parameter to this command.
%%\acmSubmissionID{123-A56-BU3}

%%
%% For managing citations, it is recommended to use bibliography
%% files in BibTeX format.
%%
%% You can then either use BibTeX with the ACM-Reference-Format style,
%% or BibLaTeX with the acmnumeric or acmauthoryear sytles, that include
%% support for advanced citation of software artefact from the
%% biblatex-software package, also separately available on CTAN.
%%
%% Look at the sample-*-biblatex.tex files for templates showcasing
%% the biblatex styles.
%%

%%
%% The majority of ACM publications use numbered citations and
%% references.  The command \citestyle{authoryear} switches to the
%% "author year" style.
%%
%% If you are preparing content for an event
%% sponsored by ACM SIGGRAPH, you must use the "author year" style of
%% citations and references.
%% Uncommenting
%% the next command will enable that style.
%%\citestyle{acmauthoryear}

%%
%% end of the preamble, start of the body of the document source.
\begin{document}

%%
%% The "title" command has an optional parameter,
%% allowing the author to define a "short title" to be used in page headers.
\title{Sketching Design Requirements for Collaboration Analytics for Hackathons}

%%
%% The "author" command and its associated commands are used to define
%% the authors and their affiliations.
%% Of note is the shared affiliation of the first two authors, and the
%% "authornote" and "authornotemark" commands
%% used to denote shared contribution to the research.
\author{Ben Trovato}
\authornote{Both authors contributed equally to this research.}
\email{trovato@corporation.com}
\orcid{1234-5678-9012}
\author{G.K.M. Tobin}
\authornotemark[1]
\email{webmaster@marysville-ohio.com}
\affiliation{%
  \institution{Institute for Clarity in Documentation}
  \streetaddress{P.O. Box 1212}
  \city{Dublin}
  \state{Ohio}
  \country{USA}
  \postcode{43017-6221}
}

\author{Lars Th{\o}rv{\"a}ld}
\affiliation{%
  \institution{The Th{\o}rv{\"a}ld Group}
  \streetaddress{1 Th{\o}rv{\"a}ld Circle}
  \city{Hekla}
  \country{Iceland}}
\email{larst@affiliation.org}

\author{Valerie B\'eranger}
\affiliation{%
  \institution{Inria Paris-Rocquencourt}
  \city{Rocquencourt}
  \country{France}
}

\author{Aparna Patel}
\affiliation{%
 \institution{Rajiv Gandhi University}
 \streetaddress{Rono-Hills}
 \city{Doimukh}
 \state{Arunachal Pradesh}
 \country{India}}

\author{Huifen Chan}
\affiliation{%
  \institution{Tsinghua University}
  \streetaddress{30 Shuangqing Rd}
  \city{Haidian Qu}
  \state{Beijing Shi}
  \country{China}}

\author{Charles Palmer}
\affiliation{%
  \institution{Palmer Research Laboratories}
  \streetaddress{8600 Datapoint Drive}
  \city{San Antonio}
  \state{Texas}
  \country{USA}
  \postcode{78229}}
\email{cpalmer@prl.com}

\author{John Smith}
\affiliation{%
  \institution{The Th{\o}rv{\"a}ld Group}
  \streetaddress{1 Th{\o}rv{\"a}ld Circle}
  \city{Hekla}
  \country{Iceland}}
\email{jsmith@affiliation.org}

\author{Julius P. Kumquat}
\affiliation{%
  \institution{The Kumquat Consortium}
  \city{New York}
  \country{USA}}
\email{jpkumquat@consortium.net}

%%
%% By default, the full list of authors will be used in the page
%% headers. Often, this list is too long, and will overlap
%% other information printed in the page headers. This command allows
%% the author to define a more concise list
%% of authors' names for this purpose.
\renewcommand{\shortauthors}{Trovato and Tobin, et al.}

%%
%% The abstract is a short summary of the work to be presented in the
%% article.
\begin{abstract}
  A clear and well-documented \LaTeX\ document is presented as an
  article formatted for publication by ACM in a conference proceedings
  or journal publication. Based on the ``acmart'' document class, this
  article presents and explains many of the common variations, as well
  as many of the formatting elements an author may use in the
  preparation of the documentation of their work.
\end{abstract}

%%
%% The code below is generated by the tool at http://dl.acm.org/ccs.cfm.
%% Please copy and paste the code instead of the example below.
%%
\begin{CCSXML}
<ccs2012>
 <concept>
  <concept_id>00000000.0000000.0000000</concept_id>
  <concept_desc>Do Not Use This Code, Generate the Correct Terms for Your Paper</concept_desc>
  <concept_significance>500</concept_significance>
 </concept>
 <concept>
  <concept_id>00000000.00000000.00000000</concept_id>
  <concept_desc>Do Not Use This Code, Generate the Correct Terms for Your Paper</concept_desc>
  <concept_significance>300</concept_significance>
 </concept>
 <concept>
  <concept_id>00000000.00000000.00000000</concept_id>
  <concept_desc>Do Not Use This Code, Generate the Correct Terms for Your Paper</concept_desc>
  <concept_significance>100</concept_significance>
 </concept>
 <concept>
  <concept_id>00000000.00000000.00000000</concept_id>
  <concept_desc>Do Not Use This Code, Generate the Correct Terms for Your Paper</concept_desc>
  <concept_significance>100</concept_significance>
 </concept>
</ccs2012>
\end{CCSXML}

\ccsdesc[500]{Do Not Use This Code~Generate the Correct Terms for Your Paper}
\ccsdesc[300]{Do Not Use This Code~Generate the Correct Terms for Your Paper}
\ccsdesc{Do Not Use This Code~Generate the Correct Terms for Your Paper}
\ccsdesc[100]{Do Not Use This Code~Generate the Correct Terms for Your Paper}

%%
%% Keywords. The author(s) should pick words that accurately describe
%% the work being presented. Separate the keywords with commas.
\keywords{Do, Not, Us, This, Code, Put, the, Correct, Terms, for,
  Your, Paper}

\received{20 February 2007}
\received[revised]{12 March 2009}
\received[accepted]{5 June 2009}

%%
%% This command processes the author and affiliation and title
%% information and builds the first part of the formatted document.
\maketitle

\section{Introduction}
Hackathons are time-bounded collaborative events that have become popular for people to join forces to solve problems. Hackathons started in the early 2000s as a way for developers to work on software projects for prizes, jobs, and pizza ==[refs]==. In recent years, the collaborative notions of people coming together to solve problems have expanded hackathons to different contexts, from science, entrepreneurship, government and non-government organisations, health, and education ==[refs]==. 

Hackathons, by definition, are collaborative events, and people usually work in small independent teams~\cite{trainer2016hackathon}. Group work is studied across different research domains that include computer-supported cooperative work (CSCW), psychology, education, organisational sciences, and economics, to name a few. However, hackathons can have different traits compared to more organized group work, including the time-bounded nature and the fact that team members might meet each other for the first time. These factors mean that the teams have to establish how they collaborate rapidly, and the support people (mentors) need to be able to deal with the diversity of approaches and projects. ~\cite{edmondson2012teaming,falk2022supporting}

Organising, managing, and understanding how to create a successful hackathon is challenging for organisers. Additionally, for researchers, the complex emerging nature of these events presents interesting research opportunities for investigating emerging group work patterns, especially in CSCW. In general, hackathons have multiple teams at a scale where it is not possible to follow all the teams throughout the event to understand the different patterns of human interactions that the team members engage in when working on their projects. Therefore, teamwork and collaboration happen that can be observed by researchers. At the same time, surveys and the study of log files (chat, repository, and document histories) only provide narrow insight into what happens between team members. 

These challenges lead us to the following overall research aim:\textit{How do we study collaboration when we are not there?} Our approach is to iteratively design technologies to capture collaboration that allows us to focus on the human phenomena of cooperation and collaboration rather than what type of data different sensors can provide out of the context of how we as humans interact. This more holistic approach leads to the following two research questions: 
\begin{itemize}
    \item How do we iteratively design different technologies to capture how small groups interact in hackathons?
    \item How do we visualize the modalities of these group interactions with the aim of a future dashboard?
\end{itemize}
This short paper raises questions on ways to explore research approaches to design and rapidly prototype a collaboration analytics platform for collecting, analysing, and making sense of data for group work in hackathons and other similar situations. The paper presents the different technologies and approaches we are investigating through different iterative steps to elicit design requirements for the platform. The initial results intend to raise challenges and questions for the community to discuss what is important to capture and how to provide information from the system for different stakeholders. The paper is organized as follows, and the next section is a brief background, followed by our methodological design approach. We then introduce the context of the hackathon we have investigated and the technology and data. We then present some initial findings from the platform, followed by challenges and questions.

\section{Background}
We define hackathons as time-bounded participant events that are organised to achieve specific objectives and goals. Organizers plan the event to support these objectives through scaffolded events that unfold over the allotted time. Generally, people participating in the event have different backgrounds, expertise, and goals. The motivation to participate in these events is interest-driven to work on a shared project; however, additional incentives like prizes, networking, and community are strong drivers. During the event, the different teams create artifcacts, like software and hardware prototypes. These documents have slides and media that support new concepts that can be entrepreneurial, policies, and other socially driven ideas. The hackathon encourages innovation and novelty and pushes the participants to learn outside their expertise with big ideas ==[refs]==.  Researching these events is complex, and since they span multiple domains, it is challenging to combine the different research. 

CSCW
Technologies
Gap

\section{Methodological Approach - Design Approach}
We are guided by a broad design approach that uses the pragmatic approaches of sketching with technology and agile software development.  Since we are just starting the platform's development, we focused on 1) Awareness of the problem, 2) Suggestions, 3) Development, and 4) Evaluation through a series of different sprints or cycles.  These different iterations start identifying the different requirements for the platform \cite{Gregor20}. This paper presents the current iteration that we loosely evaluated at a hackathon. This version of the platform has been the fourth iteration, and more details can be found about the iterative cycles in previous research ==[blinded and blinded]==.

\subsection{Setting}
G48 - Idea Garage

\subsection{Data collection}

\subsection{Data analysis}

\section{Technology and Data}
mBOX
Data Pipeline
Qualitative

\section{Initial Results}
Badges
Audio
Transcripts
Observations
Interviews

\section{Challenges and Questions}
Physical Space
Patterns of Conversations
Discourse analysis
Observation
Dashboard



\section{Acknowledgments}
Remember the ACKS requirements

\bibliographystyle{ACM-Reference-Format}
\bibliography{mbib}


\end{document}
\endinput
%%
%% End of file `sample-manuscript.tex'.
